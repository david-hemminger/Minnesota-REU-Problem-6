\documentclass[11pt]{amsart}


%The package below let's us add todos to the document in a way that will stand out at a glance.  It will also list all of the current things to do, with their respective page numbers on the first page of the document.  You can add a todo like this:
%
%		Text text text \todo{change this text to something useful!}
%
%Among other things, you can edit the color of a todo:
%
%		\todo[color=\ltblue]{this todo will have a light blue background!}
%
%And you can include the todo as a line in the text, rather than in the margins:
%
%		\todo[inline]{this will appear in the middle of the page}

\usepackage[colorinlistoftodos, textsize=tiny]{todonotes}
\def\ltblue{blue!20!white}
\def\ltgreen{green!20!white}

\usepackage{amsmath,amssymb,amsthm, MnSymbol}
\usepackage{enumerate}


\newtheorem{thm}{Theorem}[section]
\newtheorem{lem}[thm]{Lemma}
\newtheorem{prop}[thm]{Proposition}
\newtheorem{cor}[thm]{Corollary}
\newtheorem{conj}[thm]{Conjecture}

\usepackage[all]{xy}
\theoremstyle{definition}
\newtheorem{defn}[thm]{Definition}
\newtheorem{rem}[thm]{Remark}
\newcommand\fri{s_4^{-1}}
\newcommand\fr{s_4}
\newcommand\tri{s_3^{-1}}
\newcommand\tr{s_3}
\newcommand\twi{s_2^{-1}}
\newcommand\tw{s_2}
\newcommand\oni{s_1^{-1}}
\newcommand\on{s_1}

\makeatletter
\providecommand\@dotsep{5}
\def\listtodoname{List of Todos}
\def\listoftodos{\@starttoc{tdo}\listtodoname}
\makeatother


\usepackage{amsmath}
\begin{document}

%This adds the list of todos to the first page.
\listoftodos

\begin{lem}
\label{5.2 analog}
[Proposition 5.2 Analog] The elements $t_i$, for $i$ a vertex of $\Gamma$, satisfy the relations $(R2)$ and $(R3)$.
\end{lem}

After Lemma \ref{lem:one_connected_to_k} we have left to check the relations $(R2)$ when both $i$ and $j$ are connected to $k$ and the relations $(R3)$.  Beginning with the relations $(R2)$, and following cases a-f from Corollary 2.3 in Barot and Marsh:


\begin{enumerate}[a)]
\item
\begin{enumerate}[i)]
\item $$t_it_j = s_ks_is_k^{-1}s_ks_js_k^{-1} = s_ks_is_js_k^{-1} = s_ks_js_is_k^{-1} = t_jt_i$$
\item $$t_it_j = s_is_j = s_js_i = t_jt_i$$
\end{enumerate}
\item
\begin{enumerate}[i)]
\item
\begin{align*}
t_it_jt_i &= s_ks_is_k^{-1}s_js_ks_is_k^{-1}\\
&= s_ks_is_js_ks_j^{-1}s_is_k^{-1}\\
&= s_ks_js_is_ks_is_j^{-1}s_k^{-1}\\
&= s_ks_js_ks_is_ks_j^{-1}s_k^{-1}\\
&= s_js_ks_js_is_j^{-1}s_k^{-1}s_j\\
&= s_js_ks_is_k^{-1}s_j\\
&= t_it_jt_i
\end{align*}

\begin{align*}
t_jt_k^{-1}t_it_k &= s_js_k^{-1}s_ks_is_k^{-1}s_k\\
&= s_js_i\\
&= s_is_j\\
&= t_k^{-1}t_it_kt_j\\
\end{align*}

\item $$t_it_j = s_is_ks_js_k^{-1} = s_is_j^{-1}s_ks_j = s_j^{-1}s_ks_js_i = s_ks_js_k^{-1}s_i = t_jt_i$$
\end{enumerate}
\item
\begin{enumerate}[i)]
\item $$t_it_j = s_ks_is_k^{-1}s_ks_js_k^{-1} = s_ks_is_js_k^{-1} = s_ks_js_is_k^{-1} = t_jt_i$$
\item $$t_it_j = s_is_j = s_js_i = t_jt_i$$
\end{enumerate}
\item
\begin{enumerate}[i)]
\item
\begin{align*}
t_it_jt_it_jt_i^{-1}t_j^{-1}t_i^{-1}t_j^{-1} &= s_ks_is_k^{-1}s_js_ks_is_k^{-1}s_js_ks_i^{-1}s_k^{-1}s_j^{-1}s_ks_i^{-1}s_k^{-1}s_j^{-1}\\
&=s_ks_is_k^{-1}s_js_ks_is_js_ks_j^{-1}s_i^{-1}s_js_k^{-1}s_j^{-1}s_i^{-1}s_k^{-1}s_j^{-1}\\
&=s_ks_is_k^{-1}s_ks_js_ks_is_ks_i^{-1}s_k^{-1}s_j^{-1}s_i^{-1}s_k^{-1}s_j^{-1}\\
&=s_ks_js_is_ks_is_ks_i^{-1}s_k^{-1}s_i^{-1}s_k^{-1}s_j^{-1}s_k^{-1}\\
&= e
\end{align*}

We also have
$$t_jt_k^{-1}t_it_k = s_js_k^{-1}s_ks_is_k^{-1}s_k = s_is_j = t_k^{-1}t_it_kt_j$$ 

\item
\begin{align*}
t_it_j &= s_is_ks_js_k^{-1}\\
&= s_is_j^{-1}s_ks_j\\
&= s_j^{-1}s_ks_js_i\\
&= s_ks_js_k^{-1}s_i\\
&= t_jt_i
\end{align*}
\end{enumerate}
\item
\begin{enumerate}[i)]
\item
\begin{align*}
t_it_jt_it_jt_i^{-1}t_j^{-1}t_i^{-1}t_j^{-1} &= s_ks_is_k^{-1}s_js_ks_is_k^{-1}s_js_ks_i^{-1}s_k^{-1}s_j^{-1}s_ks_i^{-1}s_k^{-1}s_j^{-1}\\
&= s_i^{-1}s_ks_is_js_i^{-1}s_ks_is_js_i^{-1}s_k^{-1}s_is_j^{-1}s_i^{-1}s_k^{-1}s_is_j^{-1}\\
&= s_i^{-1}s_ks_js_ks_js_k^{-1}s_j^{-1}s_k^{-1}s_j^{-1}s_i\\
&= e
\end{align*}
We also have
$$ t_jt_k^{-1}t_it_k = s_js_k^{-1}s_ks_is_k^{-1}s_k = s_js_i = s_is_j = t_it_j$$

\item
\begin{align*}
s_k^{-1}t_it_jt_i^{-1}t_j^{-1}s_k &= s_k^{-1}s_is_ks_js_k^{-1}s_i^{-1}s_ks_j^{-1}\\
&= s_is_ks_i^{-1}s_js_is_k^{-1}s_i^{-1}s_j^{-1}\\
&= e
\end{align*}

\end{enumerate}
\item
\begin{enumerate}[i)]
\item
\begin{align*}
s_k^{-1}t_it_jt_it_j^{-1}t_i^{-1}t_j^{-1} &= s_is_k^{-1}s_js_ks_is_k^{-1}s_j^{-1}s_ks_i^{-1}s_k^{-1}s_j^{-1}s_k\\
&= e
\end{align*}

\begin{align*}
t_it_j^{-1}t_kt_jt_i^{-1}t_j^{-1}s_k^{-1}t_j &= s_ks_is_k^{-1}s_j^{-1}s_ks_js_ks_i^{-1}s_k^{-1}s_j^{-1}s_k^{-1}s_j\\
&= s_ks_is_js_ks_j^{-1}s_k^{-1}s_ks_i^{-1}s_k^{-1}s_j^{-1}s_k^{-1}s_j\\
&= s_ks_is_js_ks_j^{-1}s_i^{-1}s_k^{-1}s_j^{-1}s_k^{-1}s_j\\
&= s_ks_is_js_ks_j^{-1}s_i^{-1}s_js_k^{-1}s_j^{-1}s_k^{-1}\\
&= s_ks_js_ks_j^{-1}s_is_i^{-1}s_js_k^{-1}s_j^{-1}s_k\\
&= e\\
\end{align*}

\item This follows from part (i) by symmetry
\end{enumerate}
\end{enumerate}



'\end{document}