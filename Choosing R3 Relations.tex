\documentclass[11pt]{amsart}


%The package below let's us add todos to the document in a way that will stand out at a glance.  It will also list all of the current things to do, with their respective page numbers on the first page of the document.  You can add a todo like this:
%
%		Text text text \todo{change this text to something useful!}
%
%Among other things, you can edit the color of a todo:
%
%		\todo[color=\ltblue]{this todo will have a light blue background!}
%
%And you can include the todo as a line in the text, rather than in the margins:
%
%		\todo[inline]{this will appear in the middle of the page}

\usepackage[colorinlistoftodos, textsize=tiny]{todonotes}
\def\ltblue{blue!20!white}
\def\ltgreen{green!20!white}

\usepackage{amsmath,amssymb,amsthm, MnSymbol}


\newtheorem{thm}{Theorem}[section]
\newtheorem{lem}[thm]{Lemma}
\newtheorem{prop}[thm]{Proposition}
\newtheorem{cor}[thm]{Corollary}
\newtheorem{conj}[thm]{Conjecture}

\theoremstyle{definition}
\newtheorem{defn}[thm]{Definition}
\newtheorem{rem}[thm]{Remark}


\makeatletter
\providecommand\@dotsep{5}
\def\listtodoname{List of Todos}
\def\listoftodos{\@starttoc{tdo}\listtodoname}
\makeatother


\begin{document}

%This adds the list of todos to the first page.
\listoftodos
\newpage

It's not immediately obvious which relations analogous to the original R3 relations we want to choose.  Here we try out several possible relations, in each case trying to prove an analogous lemma to Lemma 4.1 in Barot and Marsh's paper.



\section{Attempts}

\subsection{}
Here we test out the set of relations R3' where if $i_{a_0}, i_{a_1},\ldots,i_{a_{d-1}},i_{a_0}$ is a chordless cycle with all weights 1, then we add in the relation 
$$s_0s_1^{-1}s_2s_3^{-1}\cdots s_{d-2}^{(-1)^{d-2}} s_{d-1}s_{d-2}^{(-1)^{d-1}}\cdots s_4^{-1}s_3s_2^{-1}s_1 = s_1^{-1}s_2s_3^{-1}\cdots s_{d-2}^{(-1)^{d-2}} s_{d-1}s_{d-2}^{(-1)^{d-1}}\cdots s_4^{-1}s_3s_2^{-1}s_1s_0$$


We then have

\begin{align*}
& s_{d-1}s_0^{-1}s_1s_2^{-1}s_3\cdots s_{d-3}^{(-1)^{d-2}} s_{d-2}s_{d-3}^{(-1)^{d-3}}\cdots s_4s_3^{-1}s_2s_1^{-1}s_0\\
&= s_0^{-1}s_0s_{d-1}s_0^{-1}s_1s_2^{-1}s_3\cdots s_{d-3}^{(-1)^{d-2}} s_{d-2}s_{d-3}^{(-1)^{d-3}}\cdots s_4s_3^{-1}s_2s_1^{-1}s_{d-1}s_{d-1}^{-1}s_0\\
&= s_0^{-1}s_{d-1}^{-1}s_0s_{d-1}s_1s_2^{-1}s_3\cdots s_{d-3}^{(-1)^{d-2}} s_{d-2}s_{d-3}^{(-1)^{d-3}}\cdots s_4s_3^{-1}s_2s_1^{-1}s_{d-1}s_{d-1}^{-1}s_0\\
\end{align*}


...so I don't think this direction is going to work from here, simply because I don't see a way to flip all of the generators in the middle to their inverses.

An R3 that seems promising:

\begin{defn}
Using the notation as in Barot - Marsh, given a chordless cycle C in $\gamma$ so that all edges in C have weight 1, we define $r(a, a+1) = s_{a}s_{a+1}^{-1}s_{a+2}^{-1}\dots s_{a-2}^{-1}s_{a-1}s_{a-2}s_{a-3}\dots s_{a+1} = s_{a+1}^{-1}\dots s_{a-3}^{-1}s_{a-2}^{-1}s_{a-1}s_{a-2}\dots s_{a+1}s_{a}$.
\end{defn}

\begin{lem}
The relation r(a,a+1) for some vertex a $\in$ C implies the relation r(b, b+1) for all b $\in$ C.
\end{lem}
\begin{proof}
As in Barot-Marsh, it suffices to prove that the relation r(0, 1) implies the relation r(d-1, 0). So suppose $W_{\gamma}$ satisfies the relation r(0, 1). Then we have 
\begin{align*}
& s_{d-1}s_{0}^{-1}s_{1}^{-1}\dots s_{d-3}^{-1}s_{d-2}s_{d-3}\dots s_1s_0 \\
&= s_{0}^{-1}s_{0}s_{d-1}s_{0}^{-1}s_{1}^{-1}\dots s_{d-3}^{-1}s_{d-2}s_{d-3}\dots s_{1}s_{d-1}^{-1}s_{d-1}s_{0} \\
&= s_{0}^{-1}s_{d-1}^{-1}s_{0}s_{d-1}s_{1}^{-1}\dots s_{d-3}^{-1}s_{d-2}s_{d-3}\dots s_{1}s_{d-1}^{-1}s_{d-1}s_{0} \\
&= s_{0}^{-1}s_{d-1}^{-1}s_{0}s_{1}^{-1}\dots s_{d-3}^{-1}s_{d-1}s_{d-2}s_{d-1}^{-1}s_{d-3}\dots s_{1}s_{d-1}s_{0} \\
&= s_{0}^{-1}s_{d-1}^{-1}(s_{0}s_{1}^{-1}\dots s_{d-3}^{-1}s_{d-2}^{-1}s_{d-1}s_{d-2}s_{d-3}\dots s_{1})s_{d-1}s_{0} \\
&= s_{0}^{-1}s_{d-1}^{-1}(s_{1}^{-1} \dots s_{d-2}^{-1}s_{d-1}s_{d-2}s_{d-3}\dots s_{0})s_{d-1}s_{0} \\
&= s_{0}^{-1}s_{d-1}^{-1}(s_{1}^{-1} \dots s_{d-3}^{-1}s_{d-1}s_{d-2}s_{d-1}^{-1}s_{d-3}\dots s_{0})s_{d-1}s_{0} \\
&= s_{0}^{-1}s_{1}^{-1}\dots s_{d-3}^{-1}s_{d-2}s_{d-3}\dots s_{1}s_{d-1}^{-1}s_{0}s_{d-1}s_{0} \\
&= s_{0}^{-1}s_{1}^{-1}\dots s_{d-3}^{-1}s_{d-2}s_{d-3}\dots s_{1}s_{0}s_{d-1}s_{0}^{-1}s_{0} \\
&= s_{0}^{-1}s_{1}^{-1}\dots s_{d-3}^{-1}s_{d-2}s_{d-3}\dots s_{1}s_{0}s_{d-1} 
\end{align*}
as required. Note that line 3 is equal to 4 and line 7 is equal to line 8 since the cycle is chordless, meaning that $s_{d-1}$ commutes with every element except $s_{0}$ and $s_{d-2}$.
\end{proof}





\end{document}